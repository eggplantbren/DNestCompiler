\documentclass[letterpaper, 11pt]{article}
\usepackage{graphicx}
\usepackage{natbib}
\usepackage[left=3cm,top=3cm,right=3cm]{geometry}

\renewcommand{\topfraction}{0.85}
\renewcommand{\textfraction}{0.1}
\parindent=0cm

\newcommand{\dnest}{{\tt DNest}}

\title{Desiderata}
%\author{Brendon J. Brewer}

\begin{document}
\maketitle


I like the implementation of the BUGS language that is used in JAGS, but there
are some things about the language in STAN that are better, so those should
be used too (e.g. parameterising normal distributions using standard deviation
instead of precision, and calling it {\tt normal} instead of {\tt dnorm}).

\section{Simple Example}
Suppose we have a problem with one parameter called $\theta$ and a data value
called $x$. Let
the prior be $\theta \sim \textnormal{Uniform}(-10, 10)$ and let the likelihood
function
be $x \sim \mathcal{N}(\theta, \sigma^2)$ where $\sigma$ is a known fixed
standard deviation (a deterministic parent node in the graphical model).

For convenience from a \dnest~point of view, it makes sense to divide
the model specification into three blocks. The prior (which also defines all
named parameters), any deterministic nodes (things that are either known
a priori, or
derivable from the parameters using a given formula), and the likelihood.

This would be coded as:
\begin{verbatim}
# Note that the decimal points are my way of being
# a bit pedantic about knowing what's a double and what's an int.
prior
{
    theta ~ uniform(0., 1.)
}

deterministic
{
    sigma <- 1.
}

likelihood
{
    x ~ normal(theta, sigma)
}
\end{verbatim}



\end{document}

